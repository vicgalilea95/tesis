% !TEX root = MemoriaVGalilea.tex
% !TEX encoding = UTF-8 Unicode
% !TEX TS-program = pdflatex
% !TEX spellcheck = Spanish
%
%%%%%%%%%%%%%%%%%%%%%%%%%%
%----- VERSION: 9-10-2025
%%%%%%%%%%%%%%%%%%%%%%%%%%



%Formato artículo
%\bibitem{Aguilo}
%\textsc{F. Aguiló y A. Miralles}:
%{Consideraciones geométricas acerca del método de Newton},
%\textit{La Gaceta de la RSME} \textbf{7} (2004), n.º~1, 247--260.


%Formato libro
%\bibitem{Banach}
%\textsc{S. Banach}:
%\textit{Théorie des opérations linéaires},
%Monografie Matematyczne, Varsovia, 1932.

\begin{thebibliography}{999}

\frenchspacing

\bibitem{Aguilo}
\textsc{F. Aguiló y A. Miralles}:
{Consideraciones geométricas acerca del método de Newton},
\textit{La Gaceta de la RSME} \textbf{7} (2004), n.º~1, 247--260.


\bibitem{Al1}
\textsc{L. Ahlfords}:
\textit{Complex Analysis},
MacGraw--Hill, Nueva York, 1979.


\bibitem{Alligood}
\textsc{K. Alligood, T. Sauer y J. Yorke}:
\textit{Chaos: an introduction to dynamical systems},
Springer-Verlag,  Berlin-Heidelberg, 1997.

\bibitem{ABG}
\textsc{S. Amat, S. Busquier y J. M. Gutiérrez}:
{Geometric constructions of iterative functions to solve nonlinear equations},
\textit{J. Comput. Appl. Math.} \textbf{157} (2003), 197--205.


\bibitem{Argyros}
\textsc{I. K. Argyros y F. Szidarovszky}:
\textit{The theory and application of iteration methods},
C.R.C. Press Inc., Boca Raton, Florida, 1993.

\bibitem{ArgyGuti}
\textsc{I. K. Argyros y J. M. Gutiérrez}:
 {A unified approach for enlarging the radius of convergence for Newton's method and applications},
\textit{Nonlinear Functional Analysis and Applications} \textbf{10} (2005), 555--563.

\bibitem{Bailey}
\textsc{D. F. Bailey}:
{A Historical Survey of Solution by Functional Iteration},
\textit{Math. Magazine} \textbf{62} (1989), n.º~3, 155--166.

\bibitem{Balibrea}
\textsc{F. Balibrea, J. O. Freitas y J. Sousa Ramos}:
{Newton maps for quintic polynomials},
\textit{arXiv:math.DS/0501327} (2005),  1--17.


\bibitem{Banach}
\textsc{S. Banach}:
\textit{Théorie des opérations linéaires},
Monografie Matematyczne, Varsovia, 1932.

\bibitem{Banks}
\textsc{J. Banks, J. Brooks, G. Cairns, G. Davis y P. Stacey}:
{On Devaney's definition of chaos},
\textit{Amer. Math. Monthly} \textbf{99} (1992),  332--334.

\bibitem{Barna1} \textsc{B. Barna}: \"{U}ber die Divergenzpunkte des Newtonschen Verfahrens zur Bestimmung von
Wurzeln algebraischer Gleichungen. I, \textit{Publ. Math. Debrecen} \textbf{3} (1953), 109--118.
\bibitem{Barna2} \textsc{B. Barna}: \"{U}ber die Divergenzpunkte des Newtonschen Verfahrens zur Bestimmung von
Wurzeln algebraischen Gleichungen. II, \textit{Publ. Math. Debrecen} \textbf{4} (1956), 384--397.
\bibitem{Barna3} \textsc{B. Barna}: \"{U}ber die divergenzpunkte des Newtonschen verfahrens zur bestimmung von
wurzeln algebraischer gleichungen. III, \textit{Publ. Math. Debrecen} \textbf{8} (1961), 193--207.

\bibitem{Barna4} \textsc{B. Barna}: \"{U}ber die divergenzpunkte des Newtonschen verfahrens zur bestimmung von
wurzeln algebraischer gleichungen. IV, \textit{Publ. Math. Debrecen} \textbf{14} (1967), 91--97.

\bibitem{Barnsley}
\textsc{M. Barnsley}:
\textit{Fractals everywhere},
Academic Press, Boston, 1988.

\bibitem{Beardon}
\textsc{A. F.  Beardon}:
\textit{Iteration of rational functions},
Springer-Verlag, Nueva York, 1991.

\bibitem{Ben-Israel} \textsc{A. Ben-Israel}:
{Newton's method with modified functions},
\textit{Contemporary Mathematics} \textbf{204} (1997),  30--50.

\bibitem{be} \textsc{M.~Benito y J. J.~Guadalupe}:
{Dibujando mediante iteraciones},
\textit{Números} \textbf{42} (2000), 15--28.

\bibitem{BGL} \textsc{M.~Benito, J. M. Gutiérrez y V. Lanchares}:
{El fractal de Chicho},
\textit{Margarita Mathematica en memoria de José Javier (Chicho) Guadalupe Hernández}, 
Serv. Publicaciones Univ. La Rioja, Logroño (2001),  247--254.

\bibitem{Be1}
\textsc{W. Bergweiler}:
{Iteration of meromorphic functions},
\textit{Bull. Amer. Math. Soc.} \textbf{29} (1993), n.º~2, 151--188.

\bibitem{Blanc1}
\textsc{P. Blanchard}:
{Complex Analytic Dynamics on the Riemann sphere},
\textit{Bull. Amer. Math. Soc.} \textbf{11} (1984), n.º~1, 85--141.

\bibitem{Blanc2}
\textsc{P. Blanchard y A. Chiu}:
\textit{Complex Dynamics: an informal discussion},
Fractal Geometry and Analysis, Eds. J. Bélair  \&
S. Dubuc, Kluwer Academic Publishers (1991), 45--98.

\bibitem{Boettcher} 
\textsc{L. E. B\"ottcher}: {The principal laws of convergence of iterates and their application to Analysis}, 
\textit{Izv. Kasan. Fiz.-Mat. Obshch} \textbf{14} (1904), 155--234.

\bibitem{Branner} 
\textsc{B. Branner}: 
{The Mandelbrot set}, 
\textit{Proc. Symp. Applied Math.} (1989), 75--105.

\bibitem{Cajori}
\textsc{F. Cajori},
Historical note on the Newton-Raphson method of approximation,
\textit{Amer. Math. Monthly} \textbf{18} (1910),  29--33.

 \bibitem{CG}  
 \textsc{L. Carleson y T. Gamelin}:
 \textit{Complex Dynamics}, Springer-Verlag, Berlín-Heidelberg, 1993.

\bibitem{Casselman} \textsc{B. Casselman}:
YBC 7289, a precursor of the Euclid's Elements of Geometry,
\url{http://www.math.ubc.ca/~cass/Euclid/ybc/ybc.html}

\bibitem{Cayley1}
\textsc{A. Cayley}:
 {The Newton-Fourier imaginary problem},
\textit{Amer. J. Math.} \textbf{2} (1879), 97--97.

\bibitem{Cayley2}
\textsc{A. Cayley}:
{Application of the Newton-Fourier method to an
imaginary root  of an equation},
\textit{Quaterly J. Pure Appl. Math.} \textbf{16} (1879), 179--185.

\bibitem{Cayley3}
\textsc{A. Cayley}:
 {Sur les racines d'une équation algébrique},
\textit{Comptes Rendus Acad. Sci.} \textbf{110} (1890), 215--218.


\bibitem{Chabert}
\textsc{J. L. Chabert et al.}:%Lo pone así en la portada del libro
\textit{A History of Algorithms: from the Pebble to the Microchip},
Springer-Verlag, Berlín-Heidelberg, 1999.

\bibitem{Chandra}
\textsc{S. Chandrasekhar}:
\textit{Radiative transfer},
Dover, Nueva York, 1960.

\bibitem{Charles}
\textsc{E. D. Charles  y J. B. Tatum}: {The convergence of Newton-Raphson iteration with  Kepler's equation},
\textit{Celestial Mechanics and Dynamical Astronomy} \textbf{69} (1998), 357--372.

\bibitem{Colwell}
\textsc{P. Colwell}:
\textit{Solving Kepler's equation over three centuries},
Willmann-Bell, Inc., Richmond, VA, 1993.


\bibitem{Conway}
\textsc{B. A. Conway}: {An improved algorithm due to Laguerre  for the solution of  Kepler's equation},
 \textit{Celest. Mech.} \textbf{39} (1986), 199--211.

\bibitem{CM}  
\textsc{M. Cosnard y C.  Masse}:
{Convergence presque partout de la m\'ethode de Newton},
\textit{C. R. Acad. Sc. Paris} \textbf{297} (1983), 549--552.
 
\bibitem{CGS} \textsc{J. H. Curry, L. Garnett y D. Sullivan}: 
{On the iteration of rational functions: Computer experiments with Newton's method}, 
\textit{Commun. Math. Phys.} \textbf{91} (1983), 267--277.

\bibitem{Danby}
\textsc{J. M. A. Danby y T. M. Burkardt}: {The solution of Kepler equation I},
\textit{Celestial Mechanics} \textbf{40} (1983), 95--107.

\bibitem{DanbyIII}
\textsc{J. M. A. Danby y T. M. Burkardt}: {The solution of Kepler equation III},
\textit{Celestial Mechanics} \textbf{31} (1987), 303--312.

\bibitem{Dedieu}
\textsc{J. P. Dedieu}:
\textit{Points fixes, zéros et la Méthode de Newton},
Springer-Verlag, Berlín-Heidelberg, 2006.

\bibitem{Dennis}
\textsc{J. E. Dennis y R. B. Schnabel}:
\textit{Numerical  methods for unconstrained optimization and nonlinear equations},
Classics in Applied Mathematics, Vol. 16, SIAM, Filadelfia, 1996.

\bibitem{Devaney1}
\textsc{R. L. Devaney}:
\textit{A first course in Chaotic Dynamical Systems},
Addison-Wesley, Redwood City (CA), 1992.

\bibitem{Devaney}
\textsc{R. L. Devaney}:
\textit{An Introduction to Chaotic Dynamical Systems, Second Edition},
Westview Press, Cambridge, 2003.

\bibitem{Dickson}
\textsc{L. E. Dickson}:
\textit{Modern Algebraic Theories},
H. Sanborn and Co., Chicago, 1926.


\bibitem{Douady-Hubbard}
 \textsc{A. Douady y J. Hubbard}: 
 {On the dynamics of polynomial-like mappings}, 
 \textit{Ann. Scient. Ec. Norm. Sup. 4\textsuperscript{e} series} \textbf{18} (1985), 287--343.

\bibitem{EGHRR1} 
\textsc{J. A. Ezquerro, J. M. Gutiérrez, M. A. Hernández, N. Romero y M. J. Rubio}: 
{El método de Newton: de Newton a Kantorovich}, \textit{La Gaceta de la RSME} \textbf{13} (2010), n.º~1, 53--76.
 
\bibitem{EGHRR2}
\textsc{J. A. Ezquerro, J. M. Gutiérrez, M. A. Hernández, N. Romero y M. J. Rubio}: 
{Relaciones de recurrencia en el método de Newton-Kantorovich}, 
\textit{Contribuciones científicas en honor de Mirian Andrés Gómez}, 
 Serv. Publicaciones Univ. La Rioja, Logroño, (2010),  319--333.

\bibitem{Fagella}  
\textsc{N. Fagella}: {Invariants en din\`amica complexa},
\textit{Bull. Soc. Mat. Cat.} \textbf{23}  (2007), n.º~1, 29--51.

\bibitem{FagellaJarque}  
\textsc{N. Fagella y X. Jarque}: 
\textit{Iteración compleja y fractales},
\textit{Vicens Vives}, Barcelona,  2007.


\bibitem{Faires}
\textsc{J. D. Faires y R. L. Burden}:
\textit{Métodos Numéricos, 3\textsuperscript{a} Ed.}, Thomson, Madrid, 2004.

\bibitem{Fat1}
\textsc{P. Fatou}:
{Sur les équations fonctionelles},
\textit{Bull. Soc. Math. France} \textbf{47} (1919), 161--271.

\bibitem{Fat2}
\textsc{P. Fatou}:
{Sur les équations fonctionelles},
\textit{Bull. Soc. Math. France} \textbf{48} (1920), 208--314.

\bibitem{FrameMandel}  
\textsc{M. Frame y B. B. Mandelbrot}: 
\textit{Fractals, graphics and mathematics education},
{Mathematical Association of America}, Washington, DC,  2002.


\bibitem{Gallica}
\textsc{Gallica-Math: OEuvres complètes}:
Biblioteca numérica Gallica de la Bibliothèque Nationale de France,
\url{http://mathdoc.emath.fr/OEUVRES/}

\bibitem{Gilbert}
\textsc{W. J. Gilbert}:
The complex dynamics of Newton's method for a double root,
\textit{Computers Math. Applic.} \textbf{22}  (1991), n.º~10, 115--119.

\bibitem{Giraldo}
\textsc{A. Giraldo y M. A. Sastre}:
\textit{Sistemas dinámicos discretos y caos. Teoría, ejemplos y algoritmos},
Fundación General de la Universidad Politécnica de Madrid, Madrid,  2002.

\bibitem{Gulick}
\textsc{D. Gulick}:
\textit{Encounter with chaos},
McGraw Hill, Nueva York,  1992.

\bibitem{Guti21}
\textsc{J. M. Gutiérrez, M. A. Hernández y M. A. Salanova}:
Calculus of $n$th roots and third order iterative methods,
\textit{Nonlinear Analysis} \textbf{47}  (2001), 2875--2880.

\bibitem{GMV}
\textsc{J. M. Gutiérrez, \'A. A. Magreñán y J. L. Varona}:
The ``Gauss-Seidelization'' of iterative methods for solving nonlinear equations in the complex plane,
\textit{Appl. Math. Comput.} \textbf{218}  (2011), 2467--2479.


\bibitem{Guzman}
\textsc{M. de Guzmán, M. Á. Martín, M. Morán y M. Reyes}:
\textit{Estructuras fractales y sus aplicaciones},
Editorial Labor, Barcelona,  1993.

\bibitem{Hawkins}
\textsc{J. M. Hawkins}:
McMullen's root-finding algorithm for cubic polynomials,
\textit{Proc. Amer. Math. Soc.} \textbf{130}  (2002), n.º~9, 2583--2592.

\bibitem{Haruta}
\textsc{M. Haruta}:
{Newton's method on the complex exponential function},
\textit{Trans. Amer. Math. Soc.} \textbf{351} (1999), 2499--2513.

\bibitem{Head}
 \textsc{J. Head}:
\textit{The combinatorics of Newton's method for cubic polynomials}, Ph. D. Thesis, Cornell Univ., Ithaca (N. Y.), 1987.

\bibitem{Henrici}
\textsc{P. Henrici}:
\textit{Elements of Numerical Analysis}, John Wiley \& Sons, Inc.,
Nueva York, 1964.

\bibitem{MichelAmparo}
\textsc{M. A. Hernández y M. A. Salanova}:
\textit{La convexidad en la resolución de ecuaciones no lineales}, Servicio de Publicaciones de la Universidad de La Rioja, 1996.

\bibitem{Holmgren}
\textsc{R. A. Holmgren}:
\textit{A first course in discrete dynamical systems, second edition}, Springer-Verlag,
Berlín-Heidelberg, 1996.

\bibitem{Horton}
 \textsc{P. Horton}:
No fooling! Newton's method can be fooled,
\textit{Math. Magazine}  \textbf{80}
(2007), 383--387.

\bibitem{House}
\textsc{A. S. Householder}:
\textit{The numerical treatment of a single nonlinear equation}, McGraw-Hill,  Nueva York, 1970.

\bibitem{Hubbard-West}
 \textsc{J. H. Hubbard y B. West}:
 \textit{Differential equations: a dynamical systems approach},
Springer-Verlag, Berlín-Heidelberg, 1991.

\bibitem{HSS}
 \textsc{J. H. Hubbard, D. Schleicher y S. Sutherland}:
How to find all roots of complex polynomials by Newton's method,
\textit{Inv. Math.}  \textbf{146}
(2001), 1--33.

\bibitem{Hurley1}
 \textsc{H. Hurley}:
 {Multiple attractors in Newton's method},
\textit{Erg. Theory and Dyn. Systems}  \textbf{6}
(1984), 561--569.


\bibitem{Ju}
\textsc{G. Julia}:
{Memoire sur l'iteration des fonctions rationelles},
\textit{J. de Math. pures et appliqu\'ees}  \textbf{8:1}
(1918), 47--215.

\bibitem{Kalantari} \textsc{B. Kalantari}: \textit{Polynomial root-finding and polyomiography},
World Scientific Publishing Co. Pte. Ltd., Singapore, 2009.

\bibitem{Kan1}
\textsc{L. V. Kantorovich}:
On Newton's method for functional equations,
\textit{Dokl Akad. Nauk SSSR} \textbf{59} (1948), 1237--1240 (en ruso).

\bibitem{Kan2}
\textsc{L. V. Kantorovich}:
The majorant principle and Newton's method,
\textit{Dokl Akad. Nauk SSSR} \textbf{76} (1951), 17--20 (en ruso).

 \bibitem{Kanto}
\textsc{L. V. Kantorovich}:
\textit{Functional Analysis  in Normed Spaces},
Pergamon Press, Oxford, 1964.

 \bibitem{Kant-Aki}
\textsc{L. V. Kantorovich y G. P. Akilov}:
\textit{Functional Analysis},
Pergamon Press, Oxford, 1982.


 \bibitem{Kelley}
\textsc{C. T.  Kelley}:
\textit{Iterative methods for linear and nonlinear equations},
Frontiers in Applied Mathematics, Vol. 16, SIAM, Filadelfia, 1995.

\bibitem{Keller}
 \textsc{H. B. Keller}:
\textit{Numerical solution of two point boundary value problem}, SIAM,
Filadelfia, 1976.

\bibitem{Kincaid}
\textsc{D. Kincaid y W. Cheney}:
\textit{Análisis Numérico. Las matemáticas del cálculo científico},
Addison-Wesley Iberoamericana, Wilmington, 1994.

\bibitem{King}
\textsc{R. F. King}:
{Improving the Van de Vel root-finding method Algorithm},
\textit{Computing} \textbf{30} (1983),   373--378.

\bibitem{Kneisl}
\textsc{K. Kneisl}:
Julia sets for the super-Newton method, Cauchy's method and Halley's method,
\textit{Chaos}  \textbf{11} (2001), n.º~2, 359--370.

\bibitem{Knill}
\textsc{R. J. Knill}:
{A Modified Babylonian Algorithm},
\textit{Amer. Math. Monthly} \textbf{99} (1992),   734--737.

\bibitem{Kollerstrom}
\textsc{N. Kollerstrom}:
Thomas Simpson and `Newton's method of approximation': an enduring myth,
\textit{British J. Hist.  Science} \textbf{25} (1992), 347--354.

\bibitem{Koenigs} 
\textsc{G. Königs}: {Recherches sur les \`equationes fontionelles},
\textit{Ann. l'Ecole Norm.} \textbf{1} (1884), Suplement.

\bibitem{Kravanja}
\textsc{P. Kravanja y A. Haegemans}:
{A modification of Newton's method for analytic mappings having multiple zeros},
\textit{Computing} \textbf{62} (1999),  129--145.

\bibitem{Lancha}
\textsc{V. Lanchares y I. L. Pérez-Barrón}:
The dynamics of Kepler equation, en
\textit{Analytic and numerical techniques in orbital dynamics, Monogr. Real Acad. Ci. Exact. Fís.-Quím. Nat. Zaragoza} \textbf{22} (2002), 75--82.

\bibitem{Lan}
\textsc{P. Lancaster y L. Rodman}:
\textit{Algebraic Riccati equations},
Oxford University Press, Oxford, 1995.

\bibitem{La}
\textsc{M. S. Lattès}:
{Sur l'iteration de Substitutions Rationelles et Fonctions de Poincaré},
Comptes Rendus Acad. Sci. \textbf{166} (1918), 26--28.

\bibitem{Lauben}
\textsc{R. Laubenbacher, G. McGrath y D.  Pengelley}:
 Lagrange and the solution of numerical equations,
\textit{Historia Math.} \textbf{28} (2001),  \textbf{3}, 220--231.

\bibitem{Li-Yorke}
\textsc{T. Li y J. Yorke}:
Period three implies chaos,
\textit{Amer. Math. Monthly} \textbf{82}
(1975), 985--993.

\bibitem{Lucas}
\textsc{F. Lucas}:
Sur une application de la Mécanique rationnelle à la théorie des équations,
\textit{C. R. Hebdomadaires Séances Acad. Sci.} \textbf{89}
(1879), 224--226.

\bibitem{Mandelbrot}
\textsc{B. Mandelbrot}:
\textit{The fractal geometry of Nature}, W. H. Freeman and Co., Nueva York, 1982.

\bibitem{Martelli}
\textsc{M. Martelli}:
\textit{Introduction to discrete dynamical systems and chaos},
Wiley-Interscience Publ., Nueva York,
1999.

\bibitem{Mathews}
\textsc{J. H. Mathews}:
Bibliography for Newton's method,
\url{http://math.fullerton.edu/mathews/n2003/Newton'sMethodBib.html}

\bibitem{Matlab}
\textsc{Matlab}:
Repositorio del entorno de programación Matlab,
\url{http://www.mathworks.es/products/matlab/}

\bibitem{May}
 \textsc{R. May}:
   Simple mathematical models with very
complicated dynamics,
\textit{Nature} \textbf{261} (1976), 459--467.

\bibitem{McClure} 
\textsc{M. McClure}: 
{Newton's method for complex polynomials},
\textit{Mathematica in Education and Research} \textbf{11} (2006), n.º~2, 2--15.

\bibitem{McMullen}
\textsc{C. McMullen}:
 Families of rational maps and iterative root-finding algorithms,
\textit{Annals of Mathematics} \textbf{125} (1987), 467--493.

\bibitem{McMullen2}
\textsc{C. McMullen}:
\textit{Complex dynamics and renormalization},
Annals of Mathematics Studies \textbf{135}, Princeton University Press, Princeton, New Jersey, 1994.

 \bibitem{McNamee}
\textsc{J. M. McNamee}:
A bibliography on roots of polynomials: Newton's method,
\url{http://www1.elsevier.com/homepage/sac/cam/mcnamee/02.htm}

\bibitem{Milnor2006}
 \textsc{J. Milnor}:
\textit{Dynamics in one complex variable: Introductory lectures. Third edition},
Princeton University Press, Princeton, New Jersey, 2006.


\bibitem{MNTU}
 \textsc{S. Morosawa, Y. Nishimura, M. Taniguchi y T. Ueda}:
\textit{Holomorphic Dynamics},
Cambridge University Press, Cambridge, 2000.

\bibitem{Netlib}
\textsc{Netlib}:
Software repository at the University of Tennessee,
\url{http://www.netlib.org/minpack/}

\bibitem{NeuSachs}
\textsc{O. Neugebauer y A. Sachs}:
\textit{Mathematical cuneiform texts},
American Oriental Society, New Haven, Conn., 1945.


\bibitem{Nishizawa}
 \textsc{K. Nishizawa y M. Fujimura}:
 Families of rational
maps and convergence basins of Newton's method,
\textit{Proc. Japan Acad.} \textbf{68} Ser. A, (1992), 143--147.



 \bibitem{9capitulos}
\textsc{J. J. O'Connor y E. F. Robertson}:
The MacTutor History of Mathematics archive: Nine Chapters on the Mathematical Art,
\url{http://www.gap-system.org/~history/HistTopics/Nine_chapters.html}

 \bibitem{Octave}
\textsc{Octave}:
Repositorio del programa libre de cálculo numérico Octave,
\url{http://www.gnu.org/software/octave/}

\bibitem{Ort-Rh}
\textsc{J. M. Ortega y W. C.  Rheinboldt}:
\textit{Iterative solution of nonlinear equations in several variables},
 Academic Press, Nueva York, 1970.

\bibitem{Ostr2}
\textsc{A. Ostrowski}:
 Über die Konvergenz und die Abrundungsfestigkeit des Newtonschen Verfahrens,
\textit{Rec. Math.} \textbf{2} (1937), 1073--1095.

\bibitem{Ostr3}
 \textsc{A. Ostrowski}:
Über einen Fall der Konvergenz des Newtonschen Näherungsverfahrens,
\textit{Rec. Math.} \textbf{3} (1938), 254--258.

\bibitem{Ost4}\textsc{A. Ostrowski}:
\textit{Solution of equations and systems of equations},
 Academic Press, Nueva York, 1966.

\bibitem{Palacios}
\textsc{M. Palacios}:
Kepler equation and accelerated Newton method,
\textit{J. Comput. Appl. Math.} \textbf{138} (2002), 335--346.

\bibitem{Peitgen}
\textsc{H. O. Peitgen y P.H. Richter}:
\textit{The beauty of fractals},
Springer-Verlag, Berlín-Heidelberg, 1986.

 \bibitem{Peitgen2}
\textsc{H. O. Peitgen, D. Saupe y H. Jürgens}:
\textit{Fractals on the classroom. Vol. I: Introduction to fractals and chaos.},
Springer-Verlag, Nueva York, 1992.

  \bibitem{PETSc}
\textsc{PETSc}:
Portable, Extensible Toolkit for Scientific Computation,
\url{http://www.mcs.anl.gov/petsc/petsc-as/}

\bibitem{Plaza}
\textsc{S. Plaza}:
\textit{Fractales y generación computacional de imágenes},
Monografía número 16,  Instituto de Matemáticas y Ciencias Afines, IMCA, Perú, 2000.

\bibitem{PlazaRomero}
\textsc{S. Plaza y N. Romero}:
Attracting cycles for the relaxed Newton's method,
\textit{J. Comput. Appl. Math.} \textbf{235} (2011), 3238--3244.

\bibitem{Plaza-Vergara}
\textsc{S. Plaza y V. Vergara}:
Existence of periodic orbit for Newton method,
\textit{Scientia, Series A, Mathematical Sciences} \textbf{7} (2001), 31--36.


\bibitem{Polyak}
\textsc{B. T. Polyak}:
 Newton-Kantorovich method and its global convergence,
\textit{J. Math. Sciences} \textbf{133} (2006), n.º~4, 1513--1523.

\bibitem{Po-Pt2}
\textsc{F. A. Potra y V. Pták}:
\textit{Nondiscrete induction and iterative processes},
 Pitman, Londres, 1984.

\bibitem{Rall66}
\textsc{L. B. Rall}:
{Convergence of Newton process to multiple solutions},
 \textit{Numer. Math.} \textbf{9} (1966), n.º~1, 23--37.
 
\bibitem{Rall}
\textsc{L. B. Rall}:
\textit{Computational solution of nonlinear operator equations},
Robert E. Krieger Publishing Company, Huntington, Nueva York, 1979.

\bibitem{Rheinboldt}
\textsc{W. C. Rheinboldt}:
{An adaptice continuation process for solving systems of nonlinear equations},
\textit{Polish Acad. Sci. Banach Center Publ.} \textbf{3} (1977), 129--142.

\bibitem{Roberts}
\textsc{G. Roberts y J. Horgan-Kobelski}:
Newton's versus Halley's method: a dynamical systems approach,
\textit{Intern. J. Bifurcation Chaos} \textbf{14} (2004), n.º~10, 3459--3475.

\bibitem{Robinson2}
\textsc{R. C. Robinson}:
 \textit{Dynamical systems: Stability, Symbolic Dynamics and Chaos. Second Edition},
CRC Press, Nueva York, 1999.

\bibitem{Robinson1}
\textsc{R. C. Robinson}:
\textit{An introduction to dynamical systems: continuous and discrete},
Pearson Prentice Hall, New Jersey, 2004.


\bibitem{Rdguez}
\textsc{F. M. Rodríguez-Vásquez}:
 \textit{Desarrollo conceptual de los métodos iterativos en la resolución de ecuaciones no lineales: un enfoque didáctico},
Tesis Doctoral, Univ. Salamanca, 2010.
 
\bibitem{RS}
\textsc{J. R\"uckert y D. Schleicher}:
{On the Newton's method for entire functions},
\textit{J. London Math. Soc.} \textbf{75} (2007), n.º~2, 659--676.

\bibitem{Sage}
\textsc{Sage}:
Repositorio del programa de cálculo matemático Sage,
\url{http://www.sagemath.org/}


\bibitem{Sanchez}
\textsc{D. A. S\'anchez}:
An alternative to the shooting method for a certain class of boundary value problems,
 \textit{Amer. Math. Monthly} \textbf{108} (2001), n.º~6,  552--555.

\bibitem{Saunder}
\textsc{G. Saunder}:
 \textit{Iteration of rational function of one complex variable and basins of attractive fixed points},
 Ph. D. Thesis, Univ. of California, Berkeley, 1984.

\bibitem{Scheid}
\textsc{F. Scheid}:
\textit{2000 Solved Problems in Numerical Analysis},
Schaum's Solved Problem Series, McGraw--Hill,  Nueva York, 1990.

\bibitem{Schleicher}
\textsc{D. Schleicher}:
{Newton's method as a dynamical system: efficient root finding of polynomial and the Riemann $\zeta$ function},
\textit{Fields Institute Communications} \textbf{53} (2008), 1--12.


\bibitem{Schroder}
\textsc{E. Schröder}:
{Über unendlich viele Algorithmen zur
Auflösung der Gleichungen.},
\textit{Math. Ann.} \textbf{2} (1870), 317--365.
(Traducido por G. W. Stewart como \emph{On Infinitely Many Algorithms for
 Solving Equations} en 1992 (revisado en enero de 1993), disponible
 vía ftp en \url{ftp://thales.cs.umd.edu} en el directorio \texttt{pub/reports}.)
 
 \bibitem{Scilab}
\textsc{Scilab}:
The Scilab Consortium,
\url{http://www.scilab.org/}


\bibitem{Shaw}
 \textsc{W. T. Shaw}:
\textit{Complex Analysis with Mathematica},
Cambridge University Press, Cambridge, 2006.

\bibitem{Shishikura}
\textsc{M. Shishikura}: 
The connectivity of the Julia set and fixed points, en \textit{``Complex dynamics: families and friends''} 
(Ed. by D. Schleicher), A. K. Peters (2009), 257--276.


\bibitem{Smale1985}
\textsc{S. Smale}:
{On the efficiency of algorithms of analysis},
\textit{Bull. Amer. Math. Soc.} \textbf{13} (1985), n.º~4, 87--121.


\bibitem{Stewart}
\textsc{I. Stewart}:
\textit{Historia de las Matemáticas en los últimos 10000 años},
Crítica, Barcelona, 2008.

\bibitem{Strogatz}
\textsc{S. Strogatz}:
\textit{Nonlinear Dynamics and Chaos},
Addison-Wesley, Reading, MA, 1994.

\bibitem{Su}
\textsc{D. Sullivan}:
{Quasi conformal homeomorphisms and dynamics. I. Solution of Fatou-Julia problem wandering domains},
 \textit{Ann. Math.} \textbf{122} (1985), n.º~2, 401--418.

\bibitem{The2000}
\textsc{The 2000 wiewpoints Group}:
 Proof without words: geometric series,
\textit{Mathematics Magazine} \textbf{74} (2001), 320.

\bibitem{Thunberg}
\textsc{H. Thunberg}:
 Periodicity versus chaos in one--dimensional dynamics,
\textit{SIAM Review} \textbf{43} (2000), 3--30.

\bibitem{Touhey}
\textsc{P. Touhey}:
{Yet another definition of chaos},
\textit{Amer. Math. Monthly} \textbf{104} (1997),  411--415.


\bibitem{Traub}
\textsc{J. F.  Traub}:
\textit{Iterative methods for the solution of equations}, Prentice-Hall,
Englewood Cliffs, NJ, 1964.

\bibitem{VdV1}
\textsc{H. Van de Vel}:
A method for computing a root of a single nonlinear equation, including its multiplicity,
\textit{Computing} \textbf{14} (1975),  167--171.

\bibitem{VdV2}
\textsc{M. Vander Straeten y H. Van de Vel}:
Multiple root-finding methods,
\textit{J. Comput. Appl. Math.} \textbf{40} (1992), 105--114.

\bibitem{Varona}
\textsc{J. L. Varona}:
Graphic and numerical comparison between iterative methods,
\textit{Math. Intelligencer} \textbf{24} (2002), n.º~1, 37--46.

 \bibitem{YBC}
\textsc{Yale University}:
The Yale Babylonian Collection,
\url{http://www.yale.edu/nelc/babylonian.html}

\bibitem{Yam}
\textsc{T. Yamamoto}:
Historical developments in convergence analysis for Newton's and Newton-like methods,
\textit{J. Comput. Appl. Math.} \textbf{124} (2000), 1--23.

\bibitem{Yau}
\textsc{L. Yau y A. Ben-Israel}:
 The Newton and Halley methods for complex roots,
\textit{Amer. Math. Monthly}  \textbf{105} (1998), n.º~9, 806--818.

\bibitem{Ypma}
\textsc{T. J. Ypma}:
 Historical development of the Newton-Raphson method,
\textit{SIAM Review} \textbf{37} (1995), n.º~4, 531--551.

\bibitem{Walsh}
\textsc{J. Walsh}:
{The dynamics of Newton's method for cubic polynomials},
\textit{College Mathematics Journal}  \textbf{26} (1995), n.º~1, 22--28.

\bibitem{WeissBJ}
\textsc{E. W. Weisstein}:
Bring-Jerrard Quintic Form. From MathWorld, A Wolfram Web Resource.
\url{http://mathworld.wolfram.com/Bring-JerrardQuinticForm.html}


\bibitem{webZIB}
\textsc{Zuse Institute Berlin}:
Software repository for Peter Deuflhards Book
``Newton Methods for Nonlinear Problems -- Affine Invariance and Adaptive Algorithms'',
\url{http://www.zib.de/Numerik/numsoft/NewtonLib/index.en.html}


\end{thebibliography}
