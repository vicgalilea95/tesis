% TesisVG_Cap5.tex
% Capítulo 5: Estudio dinámico en la recta real
\chapter{Estudio dinámico de los métodos de Schröder, Halley y Chebyshev en el plano complejo}
\label{cap:plano_complejo}

% Resumir lo hecho en el TFM.

% \subsection{Construcci\'on del m\'etodo de Chebyshev}
% Sea $f:\mathbb{R}\to\mathbb{R}$ de clase $\mathcal{C}^3$ en un entorno de una ra\'iz simple $\alpha$ tal que $f(\alpha)=0$ y $f'(\alpha)\neq 0$. Denotamos el error $e_n=x_n-\alpha$. La idea es partir del paso de Newton y a\~nadir una correcci\'on cuadr\'atica que elimine el t\'ermino dominante del error para obtener convergencia de orden tres.

% Usando el desarrollo de Taylor de $f$ y sus derivadas alrededor de $\alpha$, escribimos
% \[
% f(x)=f'(\alpha)e+\tfrac{1}{2}f''(\alpha)e^2+\tfrac{1}{6}f^{(3)}(\alpha)e^3+\mathcal{O}(e^4),\qquad
% f'(x)=f'(\alpha)+f''(\alpha)e+\tfrac{1}{2}f^{(3)}(\alpha)e^2+\mathcal{O}(e^3),
% \]
% con $e=x-\alpha$. Si definimos
% \[
% t(x)=\frac{f(x)}{f'(x)},
% \]
% entonces una expansi\'on formal da
% \[
% t(x)=e-c_2e^2+\bigl(2c_2^2-c_3\bigr)e^3+\mathcal{O}(e^4),\quad c_2=\frac{f''(\alpha)}{2f'(\alpha)},\; c_3=\frac{f^{(3)}(\alpha)}{6f'(\alpha)}.
% \]
% El paso de Newton $x- t(x)$ produce un error $e_{n+1}=c_2e_n^2+\mathcal{O}(e_n^3)$. Para anular el t\'ermino cuadr\'atico, consideramos una correcci\'on de la forma $x- t(x)-b(x)\,t(x)^2$ y escogemos
% \[
% b(x)=\frac{1}{2}\,\frac{f''(x)}{f'(x)}.
% \]
% Con esta elecci\'on, el nuevo iterador reduce el error a orden c\'ubico, con ecuaci\'on de error
% \[
% e_{n+1}=c_3\,e_n^3+\mathcal{O}(e_n^4)=\frac{f^{(3)}(\alpha)}{6f'(\alpha)}\,e_n^3+\mathcal{O}(e_n^4).
% \]

% En forma cerrada, el \textit{m\'etodo de Chebyshev} queda
% \begin{equation}
% \label{eq:chebyshev}
%  x_{n+1} 
% 	= x_n 
% 	- \frac{f(x_n)}{f'(x_n)} 
% 	- \frac{1}{2}\,\frac{f''(x_n)}{f'(x_n)}\left(\frac{f(x_n)}{f'(x_n)}\right)^{\!2}.
% \end{equation}
% Este esquema tiene orden de convergencia tres bajo las condiciones indicadas y constante asint\'otica $\left|\tfrac{f^{(3)}(\alpha)}{6f'(\alpha)}\right|$. En comparaci\'on con Halley, requiere una evaluaci\'on adicional de $f''$, pero evita cocientes m\'as costosos como en Halley, siendo una alternativa de tercer orden basada en una correcci\'on sobre Newton que cancela el error cuadr\'atico.

% % Desarrollo ampliado basado en el PDF adjunto
% \subsubsection{Construcciones del m\'etodo de Chebyshev}

El m\'etodo de Chebyshev es un esquema iterativo de tercer orden para la
aproximaci\'on de ra\'{\i}ces simples de ecuaciones no lineales $f(x)=0$.
Su expresi\'on cl\'asica, partiendo de un $x_n$ suficientemente pr\'oximo a la ra\'{\i}z
$\alpha$, es
\begin{equation}
\label{eq:chebyshev-clasico}
 x_{n+1} 
 	= x_n 
 	- \frac{f(x_n)}{f'(x_n)} 
 	- \frac{1}{2}\,\frac{f''(x_n)}{f'(x_n)}\left(\frac{f(x_n)}{f'(x_n)}\right)^{\!2}.
\end{equation}
Este m\'etodo, junto con sus mejoras y variantes, ha sido estudiado extensamente
en la literatura debido a su buena eficiencia y a que puede deducirse de varias
maneras equivalentes.

\paragraph{Interpolaci\'on cuadr\'atica inversa.}
Una primera construcci\'on procede de aproximar la inversa local $g=f^{-1}$
en torno a $y=0$ mediante interpolaci\'on cuadr\'atica u osculadora.
Al imponer condiciones de Hermite adecuadas en $y=f(x_n)$ (valores y derivadas
que dependen de $1/f'(x_n)$ y $-f''(x_n)/f'(x_n)^3$), y evaluar en $y=0$, se
obtiene precisamente la actualizaci\'on \eqref{eq:chebyshev-clasico}.

\paragraph{Par\'abola tangente.}
Otra deducci\'on geom\'etrica recurre a una par\'abola osculadora a la gr\'afica de $f$.
Consideremos la familia
\begin{equation}
\label{eq:parabola-osculadora}
 a\,\bigl(y-f(x_n)\bigr)^2 + \bigl(y-f(x_n)\bigr) + b\,(x-x_n) = 0,
\end{equation}
cuya gr\'afica $y=y(x)$ satisface condiciones de tangencia en $x_n$:
$y(x_n)=f(x_n)$, $y'(x_n)=f'(x_n)$, $y''(x_n)=f''(x_n)$. Derivando
\eqref{eq:parabola-osculadora} y evaluando en $(x_n,f(x_n))$ se obtiene
$b=-y'(x_n)=-f'(x_n)$. Una segunda derivada y evaluaci\'on conduce a
\[ a=-\,\frac{y''(x_n)}{2\,y'(x_n)^2}=-\,\frac{f''(x_n)}{2\,f'(x_n)^2}. \]
Imponiendo $y=0$ (intersecci\'on con el eje $OX$) en \eqref{eq:parabola-osculadora}
y despejando $x$, resulta
\[
 x_{n+1}=x_n-\frac{f(x_n)}{f'(x_n)}-\frac{1}{2}\,\frac{f''(x_n)}{f'(x_n)}\left(\frac{f(x_n)}{f'(x_n)}\right)^{\!2},
\]
que coincide con \eqref{eq:chebyshev-clasico}.

\paragraph{Interpolaci\'on exponencial.}
Una tercera v\'ia de construcci\'on utiliza una curva aproximante de la forma
\begin{equation}
\label{eq:interpolacion-exp}
 y(x) = e^{a(x-x_n)}\,\bigl(b\,(x-x_n)+c\bigr),
\end{equation}
con par\'ametros $a,b,c$ escogidos para satisfacer condiciones de osculaci\'on en $x_n$:
$y(x_n)=f(x_n)$, $y'(x_n)=f'(x_n)$ y $y''(x_n)=f''(x_n)$. De la primera condici\'on
se deduce $c=f(x_n)$ y de la segunda $b=f'(x_n)-a\,f(x_n)$. La tercera condici\'on
impone la relaci\'on $2a\,f'(x_n)-a^2 f(x_n)=f''(x_n)$. Al intersectar \eqref{eq:interpolacion-exp}
con $y=0$ (esto es, imponiendo $b\,(x-x_n)+c=0$) se obtiene la actualizaci\'on
\[
 x_{n+1}=x_n-\frac{c}{b}=x_n-\frac{f(x_n)}{f'(x_n)-a\,f(x_n)}.
\]
La elecci\'on de $a$ que anula el t\'ermino cuadr\'atico del error (esto es, que asegura
convergencia c\'ubica) conduce de nuevo a la forma \eqref{eq:chebyshev-clasico}.

\medskip
Estas construcciones muestran que el m\'etodo de Chebyshev puede interpretarse
como una correcci\'on de Newton que cancela el t\'ermino de error de segundo orden
por medio de informaci\'on de segunda derivada. Bajo supuestos habituales
($f\in\mathcal{C}^3$, $f'(\alpha)\neq0$ y $x_n$ cercano a $\alpha$), el m\'etodo es de
orden tres y su ecuaci\'on de error local es
\[
 e_{n+1}=\frac{f^{(3)}(\alpha)}{6f'(\alpha)}\,e_n^3+\mathcal{O}(e_n^4).
\]



\section{Definición general y propiedades de cada método}
Los tres métodos iterativos de orden tres que se estudiarán en este capítulo son el método de Schröder, el método de Halley y el método de Chebyshev. Cada uno de ellos puede definirse mediante una función racional asociada a una función meromorfa \( f:\mathbb{C}\to\mathbb{C} \). A continuación, se presentan las definiciones y propiedades fundamentales de cada método. y los
expresarames en función del operador:
\[
 L_f(x)=\frac{f(x)f''(x)}{(f'(x))^2}.
\]
\subsection{Definición y propiedades del método de Schröder}

Sea $f:\mathbb{C}\to\mathbb{C}$ una función meromorfa. El método de Schröder asociado a $f$ es el método iterativo definido por la función racional:
\begin{equation}
	S_f(z) = z - \frac{1}{1-L_f'(z)} 
    	\frac{f(z)}{f'(z)}
\end{equation}



\subsection{Definición y propiedades del método de Halley}

Sea $f:\mathbb{C}\to\mathbb{C}$ una función meromorfa. El método de Halley asociado a $f$ es el método iterativo definido por la función racional:
\begin{equation}
	H_f(z) = z - \frac{1}{1-\frac{L_f'(z)}{2}} 
    	\frac{f(z)}{f'(z)}
\end{equation}




\subsection{Definición y propiedades del método de Chebyshev}
Sea $f:\mathbb{C}\to\mathbb{C}$ una función meromorfa. El método de Chebyshev asociado a $f$ es el método iterativo definido por la función racional:
\begin{equation}
	C_f(z) = z - \frac{f(z)}{f'(z)} \left( 1+\frac{L_f(z)}{2}\right),
\end{equation}

\section{Análisis de las funciones racionales asociadas}
\section{Estudio de los conjuntos de Julia y Fatou}
\section{Análisis del plano de parámetros}
\section{Comparación con el método de Newton en términos dinámicos}
\section{Comparación con el método de Newton en términos numéricos}
