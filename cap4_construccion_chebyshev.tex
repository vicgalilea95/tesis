\subsubsection{Construcciones del m\'etodo de Chebyshev}

El m\'etodo de Chebyshev es un esquema iterativo de tercer orden para la
aproximaci\'on de ra\'{\i}ces simples de ecuaciones no lineales $f(x)=0$.
Su expresi\'on cl\'asica, partiendo de un $x_n$ suficientemente pr\'oximo a la ra\'{\i}z
$\alpha$, es
\begin{equation}
\label{eq:chebyshev-clasico}
 x_{n+1} 
 	= x_n 
 	- \frac{f(x_n)}{f'(x_n)} 
 	- \frac{1}{2}\,\frac{f''(x_n)}{f'(x_n)}\left(\frac{f(x_n)}{f'(x_n)}\right)^{\!2}.
\end{equation}
Este m\'etodo, junto con sus mejoras y variantes, ha sido estudiado extensamente
en la literatura debido a su buena eficiencia y a que puede deducirse de varias
maneras equivalentes.

\paragraph{Interpolaci\'on cuadr\'atica inversa.}
Una primera construcci\'on procede de aproximar la inversa local $g=f^{-1}$
en torno a $y=0$ mediante interpolaci\'on cuadr\'atica u osculadora.
Al imponer condiciones de Hermite adecuadas en $y=f(x_n)$ (valores y derivadas
que dependen de $1/f'(x_n)$ y $-f''(x_n)/f'(x_n)^3$), y evaluar en $y=0$, se
obtiene precisamente la actualizaci\'on \eqref{eq:chebyshev-clasico}.

\paragraph{Par\'abola tangente.}
Otra deducci\'on geom\'etrica recurre a una par\'abola osculadora a la gr\'afica de $f$.
Consideremos la familia
\begin{equation}
\label{eq:parabola-osculadora}
 a\,\bigl(y-f(x_n)\bigr)^2 + \bigl(y-f(x_n)\bigr) + b\,(x-x_n) = 0,
\end{equation}
cuya gr\'afica $y=y(x)$ satisface condiciones de tangencia en $x_n$:
$y(x_n)=f(x_n)$, $y'(x_n)=f'(x_n)$, $y''(x_n)=f''(x_n)$. Derivando
\eqref{eq:parabola-osculadora} y evaluando en $(x_n,f(x_n))$ se obtiene
$b=-y'(x_n)=-f'(x_n)$. Una segunda derivada y evaluaci\'on conduce a
\[ a=-\,\frac{y''(x_n)}{2\,y'(x_n)^2}=-\,\frac{f''(x_n)}{2\,f'(x_n)^2}. \]
Imponiendo $y=0$ (intersecci\'on con el eje $OX$) en \eqref{eq:parabola-osculadora}
y despejando $x$, resulta
\[
 x_{n+1}=x_n-\frac{f(x_n)}{f'(x_n)}-\frac{1}{2}\,\frac{f''(x_n)}{f'(x_n)}\left(\frac{f(x_n)}{f'(x_n)}\right)^{\!2},
\]
que coincide con \eqref{eq:chebyshev-clasico}.

\paragraph{Interpolaci\'on exponencial.}
Una tercera v\'ia de construcci\'on utiliza una curva aproximante de la forma
\begin{equation}
\label{eq:interpolacion-exp}
 y(x) = e^{a(x-x_n)}\,\bigl(b\,(x-x_n)+c\bigr),
\end{equation}
con par\'ametros $a,b,c$ escogidos para satisfacer condiciones de osculaci\'on en $x_n$:
$y(x_n)=f(x_n)$, $y'(x_n)=f'(x_n)$ y $y''(x_n)=f''(x_n)$. De la primera condici\'on
se deduce $c=f(x_n)$ y de la segunda $b=f'(x_n)-a\,f(x_n)$. La tercera condici\'on
impone la relaci\'on $2a\,f'(x_n)-a^2 f(x_n)=f''(x_n)$. Al intersectar \eqref{eq:interpolacion-exp}
con $y=0$ (esto es, imponiendo $b\,(x-x_n)+c=0$) se obtiene la actualizaci\'on
\[
 x_{n+1}=x_n-\frac{c}{b}=x_n-\frac{f(x_n)}{f'(x_n)-a\,f(x_n)}.
\]
La elecci\'on de $a$ que anula el t\'ermino cuadr\'atico del error (esto es, que asegura
convergencia c\'ubica) conduce de nuevo a la forma \eqref{eq:chebyshev-clasico}.

\medskip
Estas construcciones muestran que el m\'etodo de Chebyshev puede interpretarse
como una correcci\'on de Newton que cancela el t\'ermino de error de segundo orden
por medio de informaci\'on de segunda derivada. Bajo supuestos habituales
($f\in\mathcal{C}^3$, $f'(\alpha)\neq0$ y $x_n$ cercano a $\alpha$), el m\'etodo es de
orden tres y su ecuaci\'on de error local es
\[
 e_{n+1}=\frac{f^{(3)}(\alpha)}{6f'(\alpha)}\,e_n^3+\mathcal{O}(e_n^4).
\]

